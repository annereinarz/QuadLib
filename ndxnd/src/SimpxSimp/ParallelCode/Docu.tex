\documentclass[a4paper,10pt]{article}

\title{Parallel Implementation of a Quadrature Algorithm}
\author{A. Reinarz}

\usepackage[T1]{fontenc}
\usepackage{ucs}
\usepackage[utf8x]{inputenc}
\usepackage{ngerman}
\usepackage[ngerman]{babel}
\usepackage{amsmath}
\usepackage{amssymb}
\usepackage{graphicx}
\usepackage{bbm}
\usepackage{picinpar}

\renewcommand{\labelenumi}{(\roman{enumi})}
\usepackage{geometry}
\geometry{tmargin=1cm,bmargin=2.5cm,lmargin=1cm,rmargin=1cm}
\newcommand{\real}{\mathbbm{R}}

\parindent=0pt

\begin{document}
\numberwithin{equation}{subsection}

\maketitle
\tableofcontents
\newpage

\section{Structures}
Let $S^{(1)}$ and $S^{(2)}$ denote the two simplices we want to integrate over, they are each of dimension $d$.
With $k$ we denote the dimension of $S^{(1)}\cap S^{(2)}$. 
\subsection{Vertexlist}
A structure containing a list of the vertices of the two simplices $S^{(1)}$ and $S^{(2)}$.
\begin{verbatim}
  int s1;   int s2;
  int** vtxlist;
\end{verbatim}
Here $\verb=s1= = d$, the spatial dimension and $\verb=s2= = 2d-(k+1)$.
\verb=vtxlist= contains the coordinates of the vertices, in the following order:
\begin{enumerate}
\item $k+1$ shared vertices of $S^{(1)}$ and $S^{(2)}$
\item $d-k-1$ remaining vertices of $S^{(1)}$
\item $d-k-1$ remaining vertices of $S^{(2)}$
\end{enumerate}
\subsubsection*{Important routines}
There are routines for the initialisation and deallocation of the memory for this structure, they can be 
found in main.c and are described more thoroughly in the section on routines.
\begin{itemize}
\item \begin{verbatim} Vertexlist get_vertexlist(int sim);\end{verbatim}
\item \begin{verbatim} void free_vertexlist(Vertexlist vtxlist);\end{verbatim}
\end{itemize}

\subsection{AffineTrafo}
A structure containing the parameters of the affine transformation
 $$S_d \times S_d\rightarrow S^{(1)} \times S^{(2)}$$
using the same notation as in step 1 of the coordinate transformations.
\begin{verbatim}
  double** A1;   double** A2;
  double* v10;   double* v20;
  int lvtx;      int d;
\end{verbatim}
\subsubsection*{Important routines}
The routine 
\begin{verbatim}AffineTrafo determineAffineTrafo(int d,int k, Vertexlist vtxlist);\end{verbatim}
initialises the affine transformation and sets the values using the list of vertices.

\subsection{QuadraturePoints}
A structure containing the quadrature points.
\begin{verbatim}
  int n;     int d;
  double** t;
  double* wt;
\end{verbatim}
\verb=t= contains the quadrature points, it has the size $n \times d$. \verb=wt= is a vector of length n 
containing the quadrature weights corresponding to a line of \verb=t=.
\subsubsection*{Example}
Using this structure to calculate an integral:
\begin{displaymath}
  \int_D \varphi(x) dx = \sum_i \varphi(t[i])\cdot wt[i],~~\text{ where }D\subset \real^{2d}.
\end{displaymath}

\subsubsection*{Important routines}
There are routines for the initialisation and deallocation of the memory for this structure. There are
also routines to allow a larger number of quadrature points or a higher spatial dimension. These routines can be 
found in quadpoints.c and are described more thoroughly in the section on routines.
\begin{itemize}
\item \begin{verbatim}QuadraturePoints resizequadpointsn(QuadraturePoints quadpoints, int size);\end{verbatim}
\item \begin{verbatim}QuadraturePoints resizequadpointsd(QuadraturePoints quadpoints, int size);\end{verbatim}
\item \begin{verbatim}void init_quadpoints(QuadraturePoints* qp, int n, int d); \end{verbatim}
\item \begin{verbatim}void free_qp(QuadraturePoints qp);\end{verbatim}
\end{itemize}

\subsection{Quad2}
Another structure containing quadrature points. 
\begin{verbatim}
  int luv;    int ruv;       
  int lzh;    int rzh;          
  double **u;
  double **v;
  double **zh;
  double *w;
\end{verbatim}
Here \verb=u= is a vector of quadrature points of the size $\verb=luv= \times \verb=ruv=$,
 \verb=v= is also a vector of quadrature points of the size $\verb=luv= \times \verb=ruv=$,
\verb=zh= is a vector of quadrature points of the size $\verb=lzh= \times \verb=rzh=$ and
\verb=w= is a vector of quadrature weights of the length \verb=ruv=
The only difference to \textbf{QuadraturePoints} is that the quadrature points are stored in smaller groups.
\subsubsection*{Example}
 Using this structure to calculate an integral:
\begin{displaymath}
  \int_{S_d} \int_{S_d} \varphi(u,v,u-v) du dv = \sum_i \varphi(u[i],v[i],zh[i])\cdot w[i].
\end{displaymath}

\subsubsection*{Important routines}
There are routines for the initialisation and deallocation of the memory for this structure, they can be 
found in quadpoints.c and are described more thoroughly in the section on routines.
\begin{itemize}
\item \begin{verbatim}void free_Quad2(Quad2 qp);\end{verbatim}
\item \begin{verbatim}void init_Quad2(Quad2* qp, int luv, int ruv, int lzh, int rzh);\end{verbatim}
\end{itemize}
\newpage

\section{Routines}
Each .c file contains several routines. Each subsection corresponds to a certain file, which is usually named
after the largest file it contains.
\subsection{main.c}
\subsubsection{main}
The function \verb=main= takes information on which simulation to run, sets up mpi, starts processes
 and allocates portions of the calculation to individual processes.
The information on which simulation to run must be entered into the file by hand.
The main calls \verb=get_vertexlist= and \verb=determineAffineTrafo=.
The number of processes necessary for the rest of the algorithm depends on \verb=k=, 
the dimension of intersection.
\begin{description}
\item[k=0, -1] Only one process is necessary. The main simply calls squad\textunderscore nonperm with the first process
                 and prints the returned value.
\item[k>0] $2^{k}-1$ processes necessary. To allocate portions of algorithm the first process runs over two 
for loops. The first loop runs from $j = 0$ to $k-1$, the second loop runs from $perm=0$ to 
$\frac{k!}{j!(k-j)!}$ and sends each process the number of a permutation and a j. Then each process calls 
\verb=squad_perm= and calculates part of the quadrature for its permutation and j.\\
Then the first process collects the values calculated by the other processes and sums them up.                
\end{description}
 
\subsubsection{get\textunderscore vertexlist}
\textbf{Usage}: \verb=Vertexlist get_vertexlist(int sim);=\\
This function \verb=get_vertexlist= recieves an integer \verb=sim= and creates a Vertexlist. \verb=sim= is
a two digit number the first number gives the spatial dimension \verb=d= and the second digit gives
the dimension of the intersection of the two simplices \verb=k=. Then the function allocates memory and
chooses a Vertexlist from a predefined list.
\subsubsection{free\textunderscore vertexlist}
\textbf{Usage}: \verb=void free_vertexlist(Vertexlist vtxlist);=\\
The function \verb=free_vertexlist= recieves a Vertexlist and deallocates the memory.

\subsection{functions.c}
Contains the functions we wish to integrate. So far there are four choices:
\begin{enumerate}
\item \begin{verbatim}double* Function1(Quad2 q2 , int d, int k);    \end{verbatim}
       This function is given by: $F(x,y,z)=\|z\| ^{\alpha}\text{, we set }\alpha=-2d+\epsilon.$
\item \begin{verbatim}double* Function2(Quad2 q2, int d, int k);    \end{verbatim}
       This function is given by: $F(x,y,z)=\|x(i)-y(i)\| ^{\alpha}\text{, we set }\alpha=-2d+k+\epsilon.$
\item \begin{verbatim}double* Function3(Quad2 q2, int d, int k);    \end{verbatim}
       This function is given by: $F(x,y,z)=\|z\| ^{-1}$
\item \begin{verbatim}double* Function4(Quad2 q2, int d, int k);    \end{verbatim}
       This function is given by: $F(x,y,z)=\|(x(i)-y(i)\| ^{-1}$
\end{enumerate}

\subsection{BasicSubQuad.c}
\subsubsection{BasicSubQuad}
\textbf{Usage}: \verb=void OriginQuad(int* active,int size, QuadraturePoints* qp);=\\
The function \verb=BasicSubQuad= realises steps 8,7 and 6 of the coordinate transforms and computes 
the basic subquadrature for the approximation of 
\begin{displaymath}
 \int_{S_{d-k}} \int_{S_{d-k}} \int_{A^+_{N_j,R_j}} \int_{E_k(z)} \varphi 
                     (\check{u},\check{v},\tilde{z},\tilde{u}) d\check{u}d\check{v}d\tilde{z}d\tilde{u}.
\end{displaymath}
\verb=BasicSubQuad= recieves the number of the basic subquadrature \verb=j=, the dimension of
 $S^{(1)}\cap S^{(2)}$ \verb=k=, the spatial dimension \verb=d= and a pointer to QuadraturePoints \verb=qp=. 
The QuadraturePoints \verb=qp= should contain a tensor product quadrature rule on $ [0,1]^d$.
It then writes the points and weights of the basic subquadrature into \verb=qp=. To do so it uses the 
functions \verb=OriginQuad= and \verb=C2S=.
\subsubsection{OriginQuad}
\textbf{Usage}: \verb=void C2S(QuadraturePoints* qp,int startidx, int endidx);=\\
The function \verb=OriginQuad= realises step 8 of the coordinate transformations. It recieves a pointer 
to QuadraturePoints \verb=qp=, containing a quadrature rule on a cube and a range of indices, which 
participate in the transformation. These indices run from \verb=startidx= to \verb=endidx=. It writes 
the resultant quadrature rule into \verb=qp=.
\subsubsection{C2S}
\textbf{Usage}: \verb=void BasicSubQuad( int j, int k, int d, QuadraturePoints* quadpoints);=\\
The function \verb=C2S= realises the parametrisation: 
\begin{displaymath}
  S_d=\{x=\chi(t)~:~t\in [0,1]^d\},\text{ where }\chi_j(t)=\begin{cases}t_1...t_j(1-t_{j+1})~~&j=1,...,d-1,\\
                                                                t_1...t_d&j=d.\end{cases}
\end{displaymath}
\verb=C2S= recieves a pointer to QuadraturePoints \verb=qp= and writes the resultant quadrature rule into 
\verb=qp=.

\subsection{Quadpoints.c}
\subsubsection{resizequadpointsn}
\textbf{Usage}: \verb=QuadraturePoints resizequadpointsn(QuadraturePoints qp, int size);=\\
The function \verb=resizequadpointsn= recieves QuadraturePoints \verb=qp= and an integer \verb=size=. 
It then allocates enough memory for the QuadraturePoints to be $size\cdot qp.n$ long.
\subsubsection{resizequadpointsd}
\textbf{Usage}: \verb=QuadraturePoints resizequadpointsd(QuadraturePoints qp, int size);=\\
The function \verb=resizequadpointsd= recieves QuadraturePoints \verb=qp= and an integer \verb=size=. 
It then allocates enough memory for the QuadraturePoints to have $size\cdot qp.d$ dimensions.
It is currently not needed by the code.
\subsubsection{init\textunderscore quadpoints}
\textbf{Usage}: \verb=void init_quadpoints(QuadraturePoints* qp, int n, int d); =\\
The function \verb=init_quadpoints= allocates memory for the QuadraturePoints \verb=qp=. The integers
are set as follows: $qp.n=n$ and $qp.d=d$.
\subsubsection{init\textunderscore Quad2}
\textbf{Usage}: \verb=void init_Quad2(Quad2* qp, int luv, int ruv, int lzh, int rzh);=\\
The function \verb=init_Quad2= allocates memory for the Quad2 \verb=qp=. The integers
are set as follows: $qp.luv=luv$, $qp.ruv=ruv$, $qp.lzh=lzh$ and $qp.rzh=rzh$.
\subsubsection{free\textunderscore qp}
\textbf{Usage}: \verb=void free_qp(QuadraturePoints qp);=\\
The function \verb=free_qp= recieves QuadraturePoints and deallocates the memory.
\subsubsection{free\textunderscore Quad2}
\textbf{Usage}: \verb=void free_Quad2(Quad2 qp);=\\
The function \verb=free_Quad2= recieves Quad2 and deallocates the memory.

\subsection{Tensorquad.c}\label{sec:Tensorquad}
\subsubsection{GLquad}
\textbf{Usage}: \verb=void GLquad(QuadraturePoints* qp, int n, double a, double b);=\\
The function \verb=GLquad= computes the Gauss-Lobatto quadrature rule on the interval $[\verb=a=,\verb=b=]$
 with \verb=n= quadrature points and weights. It writes the quadrature rule into \verb=qp=.
\subsubsection{CGLquad}
\textbf{Usage}: \verb=void CGLquad(QuadraturePoints* qp, int n);=\\
The function \verb=CGLquad= computes the composite-Gauss-Lobatto quadrature rule on the interval 
$[0,1]$. To do so it divides $[0,1]$ into $\verb=n=-1$ intervals and calls \verb=GLquad= on each one 
with $\lceil\verb=n=(1+(1-j)/\verb=n=)\rceil$ points, where j corresponds to the number of the interval.
 It writes the resultant quadrature rule into \verb=qp=.
\subsubsection{TensorQuad}
\textbf{Usage}: \verb=void TensorQuad(int it, int lx, QuadraturePoints*QP, QuadraturePoints qp);=\\
The function \verb=TensorQuad= generates a tensorproduct quadrature rule. It recieves a pointer to
 QuadraturePoints \verb=QP= and QuadraturePoints \verb=qp= and two integers \verb=it=
and \verb=lx=. It uses the integers to calculate the actual size of the QuadraturePoints. This is necessary
since enough space has been allocated for \verb=QP= to hold the resultant tensor product quadrature rule, and
as a result \verb=QP.n= and \verb=qp.d= do not hold the actual size of \verb=QP=. Instead they hold the size
the tensorproduct quadrature rule will have.\\
Let $ly=\verb=qp.n=$ be the number of quadrature points in \verb=qp=, $lx=lx \cdot ly^{\verb=it=-1}$, be the number of 
quadrature points in \verb=QP= and $ny=\verb=qp.d=$,$nx=it*\verb=qp.d=$ correspond to the dimension of \verb=qp= and 
\verb=QP= respectively. The resultant tensor product quadrature rule has the size $l_x l_y \times n_x+n_y$.
If the quadrature rules in \verb=QP= and \verb=qp= were suitable for the approximation of integrals of the 
form:\begin{displaymath}
      \int_{D_1}\varphi_1(x)dx\approx\sum_{i=0}^{l_x-1}\varphi_1(\verb=QP.t=[i])\verb=QP.wt=[i] \text{ and }
      \int_{D_2}\varphi_2(y)dy\approx\sum_{i=0}^{l_y-1}\varphi_2(\verb=qp.t=[i])\verb=qp.wt=[i]
     \end{displaymath}
then the resultant quadrature rule suitable for the approximation of:
\begin{displaymath}
 \int_{D_1}\int_{D_2}\varphi_3(x,y)dx dy\approx\sum_{i=0}^{l_x-1}\sum_{j=0}^{l_y-1}
                \varphi_3(\verb=QP.t=[i],\verb=qp.t=[j])\verb=QP.wt=[i]\verb=qp.wt=[j]
\end{displaymath}

\subsection{squad.c}
\subsubsection{determinant}
\textbf{Usage}: \verb=double determinant(double**A, int m);=\\
The function \verb=determinant= calculates the determinant of a given matrix \verb=A= of size \verb=m=.
It uses the Lapack function \verb=dgetrf_=.
\subsubsection{Quadrature}
\textbf{Usage}: \verb=void Quadrature(int d, int k,int whichF, double* ptrQ, Quad2 q);=\\
The function \verb=Quadrature= calculates the actual quadrature. It recieves the spatial dimension\verb=d=,
 the dimension of the intersection \verb=k=, an integer corresponding to a choice of function to integrate
\verb=whichF=, a pointer to a double \verb=ptrQ= which is used to store the result
 and a Quad2 \verb=q= containing the quadrature rule to use. 
\subsubsection{determineAffineTrafo}
\textbf{Usage}: \verb=AffineTrafo determineAffineTrafo(int d,int k, Vertexlist vtxlist);=\\
The function \verb=determineAffineTrafo= computes the parameters of the affine transformation to the
 physical coordinates, as descibed in step 1 of the coordinate transformations. It recieves the spatial
dimension \verb=d=, the dimension of the intersection \verb=k= and a list of vertices \verb=vtxlist= and returns
an AffineTrafo containing the parameters of the transformation.
\subsubsection{QuadratureRule}
\textbf{Usage}: \verb=QuadraturePoints QuadratureRule(int k, int d, int nr, int ns);=\\
The function \verb=QuadratureRule= computes the quadrature rule on $[0,1]^{2d}$. It recieves the dimension
of the intersection \verb=k=, which corresponds to the number of directions for singular (composite Gauss-
Lobatto) quadrature, the spatial dimension \verb=d=, and the number of points in each direction of regular
quadrature \verb=nr= and singular quadrature \verb=ns=. It uses the routines in section \ref{sec:Tensorquad}.
\subsubsection{CalcPerm}
\textbf{Usage}: \verb=int CalcPerm(int j, int k, int d, int* U, int perm);=\\
The function \verb=CalcPerm= generates permutations in lexiographic order. It recieves the number of a 
subquadrature \verb=j=, the dimension of the intersection \verb=k=, the spatial dimension \verb=d=, a pointer to
a vector \verb=U= containing a permutation, which is assumed to be the identity permutation and an integer 
\verb=iter= giving the number of permutations to go through. The new permutation is written into \verb=U=.
\subsubsection{Quad2RefS}
\textbf{Usage}: \verb=void Quad2RefS(int k, int d,  Quad2* qp);=\\
The function \verb=Quad2RefS= realises step 2 of the coordinate transformations. It recieves the dimension
of the intersection \verb=k=, the spatial dimension \verb=d= and a pointer to a Quad2 \verb=qp=. \verb=qp=
should contain a quadrature rule before the coordinate transformation and is used to hold the result.\\
A Quad2 structure is necessary instead of a QuadraturePoints structure to avoid subtractive cancellation.
\subsubsection{Quad2PhyS}
\textbf{Usage}: \verb=Quad2 Quad2PhyS(int k, int d, Quad2 qp, AffineTrafo A );=\\
The function \verb=Quad2PhyS= realises step 1 of the coordinate transformations. It recieves the dimension
of the intersection \verb=k=, the spatial dimension \verb=d=, a Quad2 \verb=qp= and an AffineTrafo \verb=A=.
It then calculates the transformation of the quadrature rule in \verb=qp= to the physical coordinates and writes
the transformed rule back into \verb=qp=.
\subsubsection{PermRefl}
\textbf{Usage}: \begin{verbatim} void PermRefl(int k, int d, QuadraturePoints* QP, AffineTrafo A, int* P, double* Q,
                                                                int flag, int whichF);\end{verbatim}
The function \verb=PermRefl= recieves the dimension of the intersection \verb=k=, the spatial dimension
 \verb=d=, a pointer to QuadraturePoints \verb=QP=, an AffineTrafo \verb=A=, a pointer to a vector \verb=P=
containing a permuation, a pointer to a double \verb=Q=, in which to write the quadrature value, a flag \verb=flag=
and an integer \verb=whichF= corresponding to the function we want to integrate.
The functionality of \verb=PermRefl= depends on the value of \verb=flag=.
\begin{description}
\item[flag=0] This flag is only used in the case where the intersection is empty or consists of only one vertex,
               ie: $\verb=k=\leq 0$.\\
	       In this case \verb=PermRefl= realises step 1 of the coordinate transformations, transference to
               the physical simplex.
\item[flag=1] In this case \verb=PermRefl=realises the permutations from step 5 of the coordinate transformations. 
\item[flag=2] In this case \verb=PermRefl=realises the reflections from step 4 of the coordinate transformations.
\end{description}

\subsubsection{squad\textunderscore nonperm}
\textbf{Usage}: \verb=double squad_nonperm(int whichF, AffineTrafo A,int d, int k, int nr, int ns);=\\
The function \verb=squad_nonperm= is used in the case where the dimension of the intersection \verb=k= is less
than 0. In this case no subquadrature is necessary and only one process is necessary.
The function takes an integer \verb=whichF= corresponding to the choice of function to integrate, an AffineTrafo
\verb=A=, giving the affine transformation to the physical simplex, the spatial dimension \verb=d=, the number of
quadrature points to use in each regular and singular direction, given by \verb=nr= and \verb=ns= respectively.\\
This function calculates a quadrature rule on $[0,1]^{2d}$ and transforms it into a suitable quadrature rule on
the simplices given by the affine transformation. It then calculates the quadrature value and returns it.
\subsubsection{squad\textunderscore perm}
\textbf{Usage}: \verb=double squad_perm(int whichF, AffineTrafo A,int d, int k, int nr, int ns, int j, int perm);=\\
The function \verb=squad_perm= is used when the dimension of the intersection \verb=k= is greater than 0 and
subquadrature is necessary. It calculates one portion of the subquadrature for one permutation. The values 
calculated by each process in \verb=squad_perm= must still be added to get the actual quadrature value.\\
The function recieves an integer \verb=whichF= corresponding to the choice of function to integrate, an AffineTrafo
\verb=A=, giving the affine transformation to the physical simplex, the spatial dimension \verb=d=, the number of
quadrature points to use in each regular and singular direction, given by \verb=nr= and \verb=ns= respectively,
an integer \verb=j= corresponding to the number of the subquadrature and an integer \verb=perm= corresponding
to the number of the permutation to use.\\
This function calculates a quadrature rule on $[0,1]^{2d}$ and transforms it into a suitable quadrature rule on
the simplices given by the affine transformation using \verb=PermRefl= to calculate permutations and reflections
as in step 4 and 5 of the coordinate transformation. It then calculates a quadrature value and returns it.

\end{document}
